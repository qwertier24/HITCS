\documentclass[12pt, a4paper]{article}
\usepackage{minted}
\usepackage{multirow}
\usepackage{enumerate}
\usepackage{geometry}
%\geometry{left=5cm,right=5cm,top=2.5cm,bottom=2.5cm}
\usepackage{fontspec}
\setmainfont{Times New Roman}
\usepackage{minted}
\usepackage[slantfont,boldfont]{xeCJK}
\setCJKmainfont{SimSun}
\usepackage{indentfirst}
\setlength{\parindent}{2em}
\usepackage{float}
\usepackage{titling}
\usepackage{subfigure}
\usepackage{natbib}
\usepackage{booktabs}
\usepackage{amsmath}
\setCJKmonofont{SimHei}
\input zhwinfonts

\pretitle{\begin{center}\LARGE}
\posttitle{\par\end{center}\vskip 0.5em}
\preauthor{\vspace{10cm}\begin{center}
    \large \lineskip0.5em %
    \begin{tabular}[t]{c}}
\postauthor{\end{tabular}\par\end{center}}
\predate{\begin{center}\large}
\postdate{\par\end{center}}

\renewcommand{\baselinestretch}{2}
\begin{document}

\title{{\bf\Huge Scalable Bitmap Index}}
\author{Department of Computer Science and Technology\\马玉坤\\1150310618}
\date{2017/11/18}
\maketitle
\thispagestyle{empty}
\newpage
%% \tableofcontents

\begin{abstract}
  Bitmap Index is a widely used data structure in the database. It has a very low complexity to program. There are many kinds of bitmap indexes presented by different people. This article mainly focuses on the time and space analysis of these kinds of bitmap indexes and try to offer suggestions about how to choose appropriate implementations of bitmap index. Finally this article will try to show the goodness of the combination of different implementations.
\end{abstract}

\section{Background}

Bitmap Index is presented by P. O'Neil in 1987. It was first applied in a commercial database \citep{spiegler1985storage}. In the application of database, either for scientific purposes or for commercial purposes, bitmap indexes are widely used.

The original bitmap indexes uses {\emph{Bit Vector}} to indicate the indexed attributes in the database. For example, in Table \ref{table:ordinary-table}, in the column named ``math score'', the Bit Vector for value ``A'' is {\emph{101}}, indicating that Alice has an \textbf{A}, Bob does not have an A and Dean has an \textbf{A}.

\begin{table}[H]
\centering
\caption{An Ordinary Table}
\label{table:ordinary-table}
\begin{tabular}{ccc}
\toprule
name  & math score & Chinese score \\
\midrule
Alice & A          & C             \\
Bob   & B          & A             \\
Dean  & A          & D             \\
\bottomrule
\end{tabular}
\end{table}

When we use the Bit Vectors of different attributes to do logical math, we can get different results to get the answers for so many kinds of queries. Take this as another example: We want to know the students who has an \textbf{A} in math and a \textbf{C} in Chinese. Then we can use the result of the \emph{bitwise and} of the Bit Vector of ``math A'' (which is \textbf{101}) and the Bit Vector of ``Chinese C'' (which is \textbf{100}). As the calculation below, the result will be \textbf{100} which indicates that Alice is the only one who gets an \textbf{A} in math and a \textbf{C} in Chinese.

\begin{align}
  &101\\
  \& &100\\
  = &100
\end{align}

There are tons of different

\section{Introduction}

\section{Scalable Bitmap Index}

\citep{name}
%% \renewcommand\refname{Reference}
\bibliographystyle{agsm}
\bibliography{ref}
%% \begin{thebibliography}{99}
%% \bibitem{art1}Athanassoulis M, Yan Z, Idreos S. UpBit: Scalable In-Memory Updatable Bitmap Indexing[C]//Proceedings of the 2016 International Conference on Management of Data. ACM, 2016: 1319-1332.
%% \bibitem{art2}Canahuate G, Gibas M, Ferhatosmanoglu H. Update conscious bitmap indices[C]//Scientific and Statistical Database Management, 2007. SSBDM'07. 19th International Conference on. IEEE, 2007: 15-15.
%% Wu K, Ahern S, Bethel E W, et al. FastBit: interactively searching massive data[C]//Journal of Physics: Conference Series. IOP Publishing, 2009, 180(1): 01205\bibitem{art3}3.
%% \bibitem{art4}Wu K, Otoo E J, Shoshani A. Compressing bitmap indexes for faster search operations[C]//Scientific and Statistical Database Management, 2002. Proceedings. 14th International Conference on. IEEE, 2002: 99-108.
%% \bibitem{art5}Kirk S A, Walrath D E. Accelerating database queries containing bitmap-based conditions[J]. 2016.
%% \bibitem{art6}Ricci D. Method and system for highly efficient database bitmap index processing[J]. 2004.
%% \bibitem{art7}Park S, ZhengbaoWei, Yeo J. Using Bitmap Index for Optimized Technology of Large Database[J]. Mita, 2008.

%% \end{thebibliography}
%%
\end{document}
